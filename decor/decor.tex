
\documentclass[a4paper,14pt]{extreport}

% ############################################################################
% Пакеты

\usepackage{pdflscape}																	% Альбомные страницы
\usepackage{geometry}
\usepackage{floatrow}
\usepackage{cmap}																		% Улучшенный поиск русских слов в полученном pdf-файле
\usepackage[T2A]{fontenc}																% Поддержка русских букв
\usepackage[utf8]{inputenc}																% Кодировка utf8
\usepackage[english, russian]{babel}													% Языки: русский, английский
\usepackage{pscyr}																		% Красивые русские шрифты
\usepackage{amsthm, amsfonts, amsmath, amssymb, amscd}									% Математика
\usepackage{indentfirst}																% Красная строка
\usepackage[usenames]{color}															% Цвета
\usepackage{colortbl}																	% Цвета
\usepackage{longtable, ltxtable}														% Таблицы
\usepackage{multirow, makecell, array}													% Улучшенное форматирование таблиц
\usepackage{titlesec}
\usepackage[singlelinecheck = off, center]{caption}										% Многострочные подписи
\usepackage{soul}																		% Поддержка переносоустойчивых подчёркиваний и зачёркиваний
\usepackage{cite}
\usepackage[linktocpage = true, plainpages = false, pdfpagelabels = false]{hyperref}	% Гиперссылки (hidelinks - скрывает красный прямоугольник вокруг ссылки)
\usepackage{graphicx}																	% Изображения
\usepackage{tocloft}																	% Оглавление
\usepackage{changepage}
\usepackage{xspace}
\usepackage{etoolbox}
\usepackage{enumitem}
\usepackage[singlespacing]{setspace}

% ############################################################################
% Общий стиль

\geometry{a4paper, top = 2cm, bottom = 2cm, left = 2cm, right = 2cm}

% Кодировки и шрифты
\renewcommand{\rmdefault}{ftm}		% Times New Roman
\fontsize{12pt}{13pt}\selectfont

% Выравнивание и переносы
\sloppy								% Избавляемся от переполнений
\clubpenalty=10000					% Запрещаем разрыв страницы после первой строки абзаца
\widowpenalty=10000					% Запрещаем разрыв страницы после последней строки абзаца

% Оглавление
\renewcommand{\cftchapdotsep}{\cftdotsep}

% ############################################################################
% Гиперссылки

\definecolor{linkcolor}{rgb}{0.9,0,0}
\definecolor{citecolor}{rgb}{0,0.6,0}
\definecolor{urlcolor}{rgb}{0,0,1}
\hypersetup
{
	colorlinks, linkcolor = {linkcolor},
	citecolor = {citecolor}, urlcolor = {urlcolor}
}

% ############################################################################
% Переопределение именований

\renewcommand{\abstractname}{Аннотация}
\newcommand{\alsoname}{см. также}
\renewcommand{\appendixname}{Приложение}
\renewcommand{\bibname}{Литература}
\newcommand{\ccname}{исх.}
\renewcommand{\chaptername}{Глава}
\renewcommand{\contentsname}{Содержание}
\newcommand{\enclname}{вкл.}
\renewcommand{\figurename}{Рисунок}
\newcommand{\headtoname}{вх.}
\renewcommand{\indexname}{Предметный указатель}
\renewcommand{\listfigurename}{Список рисунков}
\renewcommand{\listtablename}{Список таблиц}
\newcommand{\pagename}{Стр.}
\renewcommand{\partname}{Часть}
\newcommand{\refname}{Список литературы}
\newcommand{\seename}{см.}
\renewcommand{\tablename}{Таблица}

% ############################################################################
% Списки

\setenumerate[1]{label = {\arabic*)}}
\setenumerate[2]{label = {\arabic{enumi}.\arabic{enumii}})}
\setenumerate[3]{label = {\arabic{enumi}.\arabic{enumii}.\arabic{enumiii}})}
\setenumerate[4]{label = {\arabic{enumi}.\arabic{enumii}.\arabic{enumiii}.\arabic{enumiv}})}
\setenumerate[5]{label = {\arabic{enumi}.\arabic{enumii}.\arabic{enumiii}.\arabic{enumiv}.\arabic{enumv}})}

% ############################################################################
% Библиография

\makeatletter
\bibliographystyle{decor/utf8gost705u}	% Оформляем библиографию в соответствии с ГОСТ 7.0.5
\renewcommand{\@biblabel}[1]{#1.}	% Заменяем библиографию с квадратных скобок на точку:
\makeatother

% ############################################################################
% Прочее


% ############################################################################
% Пакеты

\usepackage{tikz}
\usepackage{pgf}
\usepackage{pgfplots}
\usepackage{scalefnt}

\usetikzlibrary{shapes, snakes, arrows, automata, positioning, calc, backgrounds, datavisualization}
\usepgfplotslibrary{patchplots}

% ############################################################################
% Окружение для рисунков

%
% \mimagebegin{Метка в image:}{Подпись}
%
% Рисунок
%
% \mimageend
%

\newcommand{\mimagecaption}{}

\newcommand{\mimagebegin}[2]
{
	\begin{samepage}
	\centering
	\refstepcounter{figure}
    \label{image:#1}

	\renewcommand{\mimagecaption}{\vbox{\centering Рисунок~\thefigure~---~#2}}
}

\newcommand{\mimageend}
{
	\\
	\medskip
	\mimagecaption
	\medskip
	\end{samepage}
}

\newcommand{\mimage}[4]
{
	\vbox
	{
		\begin{center}

			\refstepcounter{figure}
			\label{image:#1}
			
			\includegraphics[#4]{#2}

			\medskip
			{\noindent \small Рисунок~\thefigure~---~#3}

		\end{center}
	}
}

% #################################################################
% Алгоритмы

\newcommand{\prevstep}{}
\newcommand{\currentdist}{}

\newcommand{\myalgo}[1]
{
\begin{tikzpicture}

	\tikzstyle {size}		=	[minimum height = 6em,		minimum width =	9em,	text width = 9em,	text centered, inner sep = 0]
	\tikzstyle {drw}		=	[																		draw, line width = 1pt, text centered]
	\tikzstyle {if}			=	[							minimum width = 5em,	text width = 5em,	drw, diamond, aspect = 3.5, inner sep = 0pt]
	\tikzstyle {do}			=	[																		drw, size, rectangle]
	\tikzstyle {begin-end}	=	[minimum height = 3em,		minimum width = 9em,	text width = 9em,	drw, rectangle, text centered, rounded corners = 1em]
	\tikzstyle {io}			=	[																		size, rectangle]
	\tikzstyle {connector}	=	[minimum height = 2.5em,	minimum width = 2.5em,	text width = 2.5em,	drw, circle, text centered]

	\tikzstyle {dist}					=	[node distance = 7em];
	\tikzstyle {dist-connector-begin}	=	[node distance = 5.5em];
	\tikzstyle {dist-connector-end}		=	[node distance = 12em];
	\tikzstyle {dist-begin-end}			=	[node distance = 5.5em];

	\tikzstyle {mpath} = [->, very thick, line width = 2pt]

	\node [begin-end] (begin) {Начало};
	\renewcommand{\prevstep}{begin}
	\renewcommand{\currentdist}{dist-begin-end}
	#1
	\node [begin-end, dist-begin-end, below of = \prevstep] (end) {Конец};
	\draw [mpath] (\prevstep) -- (end);

\end{tikzpicture}
}

\newcommand{\myio}[2]
{
	\node [io, \currentdist, below of = \prevstep] (#1) {#2};
	\draw [drw] (#1.130) -- (#1.north east) -- (#1.-50) -- (#1.south west) -- (#1.130);
	\draw [mpath] (\prevstep) -- (#1);

	\renewcommand{\prevstep}{#1}
	\renewcommand{\currentdist}{dist}
}

\newcommand{\mydo}[2]
{
	\node [do, \currentdist, below of = \prevstep] (#1) {#2};
	\draw [mpath] (\prevstep) -- (#1);

	\renewcommand{\prevstep}{#1}
	\renewcommand{\currentdist}{dist}
}

\newcommand{\myconnector}[2]
{
	\node [connector, dist-connector-begin, below of = \prevstep] (#1-begin) {#1};
	\node [connector, dist-connector-end, right of = #2] (#1-end) {#1};
	\draw [mpath] (\prevstep) -- (#1-begin);

	\renewcommand{\prevstep}{#1-end}
	\renewcommand{\currentdist}{dist-connector-begin}
}

\newcommand{\myloopgeneric}[5]
{
	\node [if, \currentdist, below of = \prevstep] (#1) {#2};
	\draw [mpath] (\prevstep) -- (#1);
	\draw [mpath] (#1.east) -- node [anchor = west, xshift = 1em] {#4} +(1em, 0) |- (#3.east);
	\draw (#1.south) +(-1em, -1em) node [anchor = east] {#5};

	\renewcommand{\prevstep}{#1}
	\renewcommand{\currentdist}{dist}
}

\newcommand{\myloop}[3]{\myloopgeneric{#1}{#2}{#3}{Истина}{Ложь}}

% #################################################################
% Разное

% Путь к каталогу с изображениями
\graphicspath{{image/}}



% ############################################################################
% Пакеты

\usepackage{listingsutf8}

\lstloadlanguages{C}
\lstset
{
    language = C,
	breaklines,
%	columns = fullflexible,
%	flexiblecolumns,
	numbers = none,
    basicstyle = \tt\fontsize{12pt}{12pt}\selectfont,
    commentstyle = ,
    showtabs = false, 
    showspaces = false,
    showstringspaces = false,
    tabsize = 2,
    inputencoding = utf8/cp1251,
	frame = single,
	showlines = true,
	resetmargins = true
}

\newcounter{lstcon}
\renewcommand{\thelstcon}{\arabic{section}.\arabic{lstcon}}

% ############################################################################
% Окружение для листингов

%
% \mylistingbegin{Метка в listing:}{Подпись}
% \begin{lstlisting}
%
% Листинг
%
% \end{lstlisting}
% \mylistingend
%

\newcommand{\mylistingcaption}{}
\newcommand{\mylistinglabel}{}

\newcommand{\mylistingbegin}[2]
{
	\refstepcounter{lstcon}
	\renewcommand{\mylistingcaption}{\vbox{\small \centering Листинг~\thelstcon~---~#2}}
	\renewcommand{\mylistinglabel}{\label{listing:#1}}
	\begin{adjustwidth}{-\leftmargin}{\rightmargin}
}

\newcommand{\mylistingend}
{
	\mylistingcaption
	\mylistinglabel
	\end{adjustwidth}
	\medskip
}

% #################################################################
% Команды

%
% Добавление файла
%
% \mysource{Путь и имя файла}{Метка в listing:}{Подпись}
%
\newcommand{\mysource}[3]
{
	\refstepcounter{lstcon}
	\label{listing:#2}
	{
		\lstinputlisting[]{#1}
		\nopagebreak
			
		\vbox{\small \centering Листинг~\thelstcon~---~#3}
		\bigskip
	}
}



\newcommand{\TODO}[1]{\colorbox{yellow}{\bf TODO #1}}
% \newcommand{\TODO}[1]{}

\newcommand{\FIXME}[1]{\colorbox{red}{\bf FIXME #1}}
% \newcommand{\FIXME}[1]{}

\newcommand{\myfloatbase}[2]{\pgfmathprintnumber[fixed, fixed zerofill, precision = 3, use comma, #2]{#1}}
\newcommand{\myfloat}[1]{\myfloatbase{#1}{}}
\newcommand{\myfloatbf}[1]{\bf \myfloatbase{#1}{assume math mode = true}}

% ############################################################################
% Структура

\newcommand{\mytitle}[1]{
\thispagestyle{empty}

\begin{center}

	\bfseries \TODO{Организация}

\end{center}

\vfill

\begin{center}

	\Large \bfseries \TODO{Название}

\end{center}

\vfill

\begin{flushright}

	Подготовил:\par\par
	\TODO{Место работы}\par
	Акинин М.В.

\end{flushright}

\vspace{5cm}

\begin{center}

	Рязань - \the\year~г.

\end{center}

\clearpage

}

\newcommand{\mycontents}
{
	\renewcommand*\contentsname{\Large Оглавление}
	\tableofcontents
	\clearpage
}

\newcommand{\myliterature}
{
	\addcontentsline{toc}{chapter}{\bibname}
	\bibliography{/home/amv/books/biblio}
}

\newcommand{\myappendix}[3]
{
	\appendix
	\chapter{#2} \label{appendix:#1}
	\input{#3}
}

\newcommand{\mychaptervn}[1]
{
	\chapter*{#1}
	\addcontentsline{toc}{chapter}{#1}
}

\titleformat{\chapter}
     {\large\bfseries}
     {\thechapter.}
     {1em}{}
  
\titleformat{\section}
     {\normalsize\bfseries}
     {\thesection}
     {1em}{}
  
\titleformat{\subsection}
     {\normalsize\bfseries}
     {\thesubsection}
     {1em}{}

\titlespacing*{\chapter}{0pt}{-30pt}{8pt}
\titlespacing*{\section}{\parindent}{*4}{*4}
\titlespacing*{\subsection}{\parindent}{*4}{*4} 



% ############################################################################
% Организации

\newcommand{\grpz}{ОАО <<ГРПЗ>>\xspace}
\newcommand{\grpzfull}{ОАО <<Государственный Рязанский приборный завод>>\xspace}

\newcommand{\nkcvkt}{НКЦ ВКТ\xspace}
\newcommand{\nkcvktfull}{Научно-конструкторский центр видеокомпьютерных технологий\xspace}

\newcommand{\labseven}{Лаборатория № 7\xspace}
\newcommand{\labsevenfull}{Лаборатория № 7\xspace}

\newcommand{\rgrtu}{ФГБОУ ВПО <<Рязанский государственный радиотехнический университет>>\xspace}
\newcommand{\rgrtufull}{Федеральное государственное бюджетное образовательное учреждение высшего профессионального образования <<Рязанский государственный радиотехнический университет>>\xspace}

\newcommand{\fips}{ФГБУ <<Федеральный институт промышленной собственности Федеральной службы по интеллектуальной собственности, патентам и товарным знакам>> (ФГБУ <<ФИПС>> - РОСПАТЕНТ)\xspace}

% ############################################################################
% Проекции и системы координат

\newcommand{\wgs}{WGS-84\xspace}
\newcommand{\utm}{UTM\xspace}
\newcommand{\wgsutm}{\wgs/\utm}



